\documentclass{article}

\usepackage{graphicx}
\usepackage{amsmath}

\begin{document}

\title{Progress report}
\author{Lukas Zorich}

\maketitle

\section{Background}

Let $X = \{ x_1, \dotsc, x_N \}$ a collection of $N$ points. Now, lets 
suppose that the features of $x_j$ are updated, and lets call $x_j^{\star}$
the updated point. Lastly, lets define 
$X_{-j} = \{ x_1, \dotsc, x_{j-1}, x_{j+1}, \dotsc, x_N \}$.

The posterior before $x_j$ moves, and using $P(\Theta)$ as the prior is

$$P(\Theta \mid X) \propto P(X \mid \Theta)P(\Theta)$$,

where $P(X \mid \Theta) = \prod_i^N P(x_i \mid \Theta)$.

Now, after $x_j$ moves to $x_j^{\star}$, the posterior is

\begin{align}
P(\Theta \mid X_{-j}, x_j^{\star}) &\propto P(X_{-j}, x_j^{\star} \mid \Theta)
    P(\Theta) \\
&\propto P(X_{-j} \mid \Theta)P(x_j^{\star} \mid \Theta)
    P(\Theta) \\
&\propto \frac{P(X \mid \Theta)}{P(x_j \mid \Theta)}P(x_j^{\star} \mid \Theta)
    P(\Theta) \\
&\propto \frac{P(x_j^{\star} \mid \Theta)}{P(x_j \mid \Theta)}P(X \mid \Theta)
    P(\Theta) \\
\end{align}

which can be written as,

\begin{equation}
\label{eq-1}
P(\Theta \mid X_{-j}, x_j^{\star}) \propto 
\frac{P(x_j^{\star} \mid \Theta)}{P(x_j \mid \Theta)}P(\Theta \mid X)
\end{equation}

Now, if we instead of considering one data point $x_j$, we consider a batch
of $S$ data points $X_J = \{ x_j, x_{j+1}, \dotsc, x_{j + S - 1}, x_{j + S} \}$
that moved from $x_{j + k}$ to $x_{j + k}^{\star}$, for $k = 0, \dotsc, S$. And
lets define $X_{-J} = \{ x_1, \dotsc, x_{j-1}, x_{j + S + 1}, \dotsc, x_N \}$.
Then, Eq. \ref{eq-1} for the batch of $S$ data point and because the prior
$P(\Theta)$ gets canceled, can be written as,

\begin{equation}
\label{eq-2}
P(\Theta \mid X_{-J}, X_{J}^{\star}) \propto 
\frac{P(\Theta \mid X_{J}^{\star})}{P(\Theta \mid X_{-J})}P(\Theta \mid X) \\
\end{equation}.

Inspired by the work done in \cite{streaming-variational-bayes}, we assume 
that we approximate the posterior using \textbf{variational inference}. Also,
we assume that $P(\Theta)$ is an exponential family distribution for $\Theta$
with sufficient statistic $T(\Theta)$ and natural parameter $\xi_0$. We suppose
further that if $q(\Theta)$ is the approximate posterior obtained using
variational inference, then $q(\Theta)$ is also in the same exponential family
with a parameter $\xi$ such that

\begin{equation}
\label{exponential-family}
q(\Theta) \propto \text{exp}(\xi \cdot T(\Theta)).
\end{equation}

Similar to \cite{streaming-variational-bayes}, when we make this assumptions
the update in Eq. \ref{eq-2} becomes

\begin{equation}
P(\Theta \mid X_{-J}, X_{J}^{\star}) \approx 
\text{exp}([\xi - \xi_{J} + \xi_{J}^{\star}] \cdot T(\Theta))
\end{equation},

where $\xi$ is the natural parameter of $q(\Theta) \approx P(\Theta \mid X)$,
and $\xi_{J}$ and $\xi_{J}^{\star}$ corresponds to the natural parameter of 
$q_{J}(\Theta) \approx p(\Theta \mid X_{J})$ and 
$q_{J}^{\star}(\Theta) \approx p(\Theta \mid X_{J}^{\star})$ respectively.

\textbf{Using this approach, we can update the posterior when data "moves" 
without the need to go through the whole dataset, instead we just need to go 
through the data points that moved to obtain $\xi_{J}$ and $\xi_{J}^{\star}$.}

\section{Application to a simplied version of my model}
For trying the proposed approach, I'm starting only with a single GMM (I prefer
to start small).

For a GMM, the update would be

\begin{align}
P(\boldsymbol{\pi}, \boldsymbol{\mu}, \boldsymbol{\Lambda} 
    \mid X_{-J}, X_J^{\star}) 
&\approx \frac{q_{-J}^{\star}(\boldsymbol{\pi}, \boldsymbol{\mu}, 
    \boldsymbol{\Lambda})}
    {q_{-J}(\boldsymbol{\pi}, \boldsymbol{\mu}, \boldsymbol{\Lambda})}
    q(\boldsymbol{\pi}, \boldsymbol{\mu}, \boldsymbol{\Lambda}) \\
\end{align}

where 

\begin{align}
q(\boldsymbol{\pi}, \boldsymbol{\mu}, \boldsymbol{\Lambda}) 
&= q(\boldsymbol{\pi})p(\boldsymbol{\mu}, \boldsymbol{\Lambda}) \\
&= Dir(\boldsymbol{\pi} \mid \boldsymbol{\alpha})
\mathcal{N}(\boldsymbol{\mu} \mid \mathbf{m}, (\beta\boldsymbol{\Lambda})^{-1})
\mathcal{W}(\boldsymbol{\Lambda} \mid \mathbf{W}, \nu).
\end{align}

\subsection{Updates}

\subsubsection{Dirichlet}

The natural parameter for the dirichlet is:

\begin{equation}
\xi = \boldsymbol{\alpha} - 1
\end{equation},

hence, the update is:

\begin{equation}
\boldsymbol{\alpha}' = \boldsymbol{\alpha} 
- \boldsymbol{\alpha}_J 
+ \boldsymbol{\alpha}_J^{\star}
\end{equation}

\subsubsection{Normal-Wishart}

The natural parameter for the Normal-Wishart distribution is:

\begin{equation}
\xi = \begin{bmatrix}
       \beta\boldsymbol{\mu}           \\[0.3em]
       \beta \\[0.3em]
       \boldsymbol{\Lambda}^{-1} + \beta\boldsymbol{\mu}\boldsymbol{\mu}^T \\[0.3em]
       \nu + 2 + p
     \end{bmatrix}
\end{equation},

hence, the updates are:

\begin{align}
\boldsymbol{\mu}' &= \frac{1}{\beta '}(\beta\boldsymbol{\mu} 
        - \beta_J\boldsymbol{\mu}_J 
        + \beta_J^{\star}\boldsymbol{\mu}_J^{\star}) \\
\beta' &= \beta - \beta_J + \beta_J^{\star} \\
\boldsymbol{\Lambda}^{-1}' &= 
    (\boldsymbol{\Lambda}^{-1} + \beta\boldsymbol{\mu}\boldsymbol{\mu}^T)
    - (\boldsymbol{\Lambda}_J^{-1} 
        + \beta_J\boldsymbol{\mu}_J\boldsymbol{\mu}_J^T)
    + (\boldsymbol{\Lambda}_J^{-1\star} 
        + \beta_J^{\star}\boldsymbol{\mu}_J^{\star}\boldsymbol{\mu}_J^{T\star})
    - \beta'\boldsymbol{\mu}'\boldsymbol{\mu}^T' \\
\nu' &= \nu - \nu_J + \nu_J^{\star} \\
\end{align}

\section{Current progress}

During the three weeks since I sent my goals for this months I distributed
my time like this:

\begin{enumerate}
\item Week 1: Try to implement the online update in PyMC. I tried to implement
this, but I kept getting errors. I asked in the PyMC forum for help, but nobody
could help me. I think that PyMC is not flexible to do this type of work.
\item Week 2: Decided to move to variational inference, because there are
several papers that use variational inference for streaming bayesian models.
I started implementing the variational inference GMM.
\item Week 3: Finished implementing the variational inference GMM. Started
working in the online update, but I'm having some bugs, so this part is 
not finished yet.
\end{enumerate}

The current progress is the following:

\begin{itemize}
\item Implementing variational bayes GMM: 100\%.
\item Implementing online update of the posterior: 70\%. I have the code written,
but I'm having a numerical underflow bug when doing the updates.
\end{itemize}

\section{Next steps}

Finish implementing this (hopefully this weekend), and depending on the results
decide the next steps.

\bibliographystyle{acm}
\bibliography{bibliography} 
\end{document}
